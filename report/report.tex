\documentclass[11pt]{article}

\usepackage[utf8]{inputenc}
\usepackage[T1]{fontenc}
\usepackage[french]{babel}
\usepackage{cite}
\usepackage{amssymb}
\usepackage{amsmath}
\usepackage{mathrsfs}
\usepackage{graphicx}

\title{Projet Robotique Autonome\\
\textit{Robot Humanoïde Sigmaban : optimisation de la marche}}
\author{Maxime Carrere, Quentin Maouhoub, Quentin Rouxel}
\date{31/01/2013}

\begin{document}

\maketitle

\section{Introduction}

L'objet de ce projet est l'optimisation de la marche du robot Sigmaban de l'équipe 
de recherche Rhoban. Il s'agit d'un robot humanoïde possédant deux bras,
deux jambes composés d'un total de 20 servos moteurs. Il embarque également 
plusieurs capteurs accéléromètres, gyroscopes et capteurs de pression.

\section{Mise en place de l'environnement}

La vingtaine de servos articulant le robots sont tous connectés en série sur un même 
bus. Le protocole utilisé pour communiquer au travers de ce bus série est défini par le
constructeur des servos moteur \textit{Dynamixel}. Le bus série est lui même relié à un mini pc
embarqué sur le robot par connexion USB. Le mini pc embarqué fait tourner un environnement Linux
(distribution Debian) sur lequel est installé le \textit{Rhoban Server} assurant la communication
bas niveau avec les servos moteurs ainsi que l'ordonnancement des différents mouvements. Le mini pc
a été configuré pour rejoindre un point d'accès WiFi. Il est ainsi possible de contrôler le robot
à distance.

\section{Stabilisation et premiers mouvements}

La création des primitives motrices est réalisée à l'aide de l'interface graphique de Rhoban.
Elle permet à l'aide d'un système de blocs interconnectés possédant chacun sa propre fonction
de construire facilement le signal à appliquer aux actionneurs du robot. Les mouvements ainsi générés
sont ensuite transmis au serveur embarqué sur le robot qui se charge de les exécuter.\\

Une première étape est le système d'équilibrage du robot. Les articulations des pieds, des genoux 
et des hanches sont utilisées pour contre balancer les chocs et déséquilibres permanents
du robot. Ce dernier dispose de capteurs accéléromètres et gyroscopes selon des axes frontale et sagittale.
A partir de ces capteurs, des boucles de régulation proportionnels et proportionnels intégrales appliquées
aux articulations permettent de maintenir debout le robot.\\

Les mouvements sont ensuite créés, basés sur un signal sinusoïdale. Le gain et la phase du signal périodique 
sont alors modifiés puis envoyés aux différentes articulations. Ces valeurs constituent l'ensemble des paramètres
du mouvement sur lesquels la méthode d'apprentissage devra agir. Le mouvement réalisé fait piétiner le robot 
toujours debout en déplaçant sont poids d'un pied sur l'autre.

\section{Détermination d'une fonction récompense}
\section{Optimisation de la marche}
\section{Conclusion}

\section{Remerciements}
Nous remercions Grégoire Passault pour avoir fortement contribué à la grande majorité de ce projet.

\end{document}

