\documentclass[11pt]{article}

\usepackage[utf8]{inputenc}
\usepackage[T1]{fontenc}
\usepackage[french]{babel}
\usepackage{cite}
\usepackage{amssymb}
\usepackage{amsmath}
\usepackage{mathrsfs}
\usepackage{graphicx}

\title{Projet Robotique Autonome\\
\textit{Robot humanoïde : optimisation de la marche}}
\author{Maxime Carrere, Quentin Maouhoub, Quentin Rouxel}
\date{31/01/2013}

\begin{document}

\maketitle

\section{Introduction}

L'objet de se projet est l'optimisation de la marche du robot de l'équipe 
de recherche Rhoban : Sigmaban. Il s'agit d'un robot humanoïde possédant deux bras,
deux jambes composés d'un total de 20 servos moteurs.

\section{Mise en place de l'environnement}

La vingtaine de servos articulants le robots sont tous connectés en série sur un même 
bus. Le protocole utiliser pour communiqué au travers de ce bus série est défini par le
constructeur des servos moteur \textit{Dynamixel}. Le bus série est ensuite relié à un mini pc
embarqué sur le robot par connexion USB. Le mini pc embarqué fait tourné un environnement linux
(distribution debian) sur lequel est installé le \textit{Rhoban Server} assurant la communication
bas niveau avec les servos moteurs ainsi que l'odonnancement des différents mouvement. Le mini pc
a été configurer pour rejoindre un point d'accès WiFi. Il est ainsi possible de controller le robot
à distance.

\section{Stabilisation et premiers pas}

La création des primitives motrices est réalisée à l'aide de l'Interface graphique de Rhoban.
Elle permet à l'aide d'un système de blocs interconnectés possédant chacun sa propre fonction
de contruire facilement le signal appliqué aux actionneurs du robots. Les mouvements ainsi généré
sont ensuite transmis au serveurs embarqué sur le robot qui se charge de les exécuter.\\

Une première étape est le système d'équilibrage du robot. 
Les mouvements créés sont tous basés sur un signal sinudoïdale. Le gain et la phase

\section{Détermination d'une fonction récompense}
\section{Optimisation de la marche}
\section{Conclusion}

\end{document}

