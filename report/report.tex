\documentclass[11pt]{article}

\usepackage[utf8]{inputenc}
\usepackage[T1]{fontenc}
\usepackage[french]{babel}
\usepackage{cite}
\usepackage{amssymb}
\usepackage{amsmath}
\usepackage{mathrsfs}
\usepackage{graphicx}

\title{Projet Robotique Autonome\\
\textit{Robot humanoïde : optimisation de la marche}}
\author{Maxime Carrere, Quentin Maouhoub, Quentin Rouxel}
\date{31/01/2013}

\begin{document}

\maketitle

\section{Introduction}

L'objet de se projet est l'optimisation de la marche du robot de l'équipe 
de recherche Rhoban : Sigmaban. Il s'agit d'un robot humanoïde possédant deux bras,
deux jambes composés d'un total de 20 servos moteurs.

\section{Mise en place de l'environnement}

La vingtaine de servos articulants le robots sont tous connecté en série sur un même 
bus. Le protocole utiliser pour communiqué au travers de ce bus série est défini par le
constructeur des servos moteur \textit{Dynamixel}. Le bus série est ensuite relié à un mini pc
embarqué sur le robot par USB.

\section{Stabilisation et premiers pas}
\section{Détermination d'une fonction récompence}
\section{Optimisation de la marche}
\section{Conclusion}

\end{document}

